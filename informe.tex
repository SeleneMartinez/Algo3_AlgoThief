\documentclass[titlepage,a4paper]{article}

\usepackage{a4wide}
\usepackage[colorlinks=true,linkcolor=black,urlcolor=blue,bookmarksopen=true]{hyperref}
\usepackage{bookmark}
\usepackage{fancyhdr}
\usepackage[spanish]{babel}
\usepackage[utf8]{inputenc}
\usepackage[T1]{fontenc}
\usepackage{graphicx}
\usepackage{float}

\pagestyle{fancy} % Encabezado y pie de página
\fancyhf{}
\fancyhead[L]{TP1 AlgoThieft}
\fancyhead[R]{Algoritmos y Programación III - FIUBA}
\renewcommand{\headrulewidth}{0.4pt}
\fancyfoot[C]{\thepage}
\renewcommand{\footrulewidth}{0.4pt}

\begin{document}
\begin{titlepage} % Carátula
	\hfill\includegraphics[width=6cm]{logofiuba.jpg}
    \centering
    \vfill
    \Huge \textbf{Trabajo Práctico 2 — Java}
    \vskip2cm
    \Large [7507/9502] Algoritmos y Programación III\\
    Curso 02 \\ % Curso 1 para el de la tarde y 2 para el de la noche
    Segundo cuatrimestre de 2021 
    \vfill
    \begin{tabular}{ | l | l | l | } % Datos del alumno
      \hline
      Padron & Apellido y Nombre & Mail \\ \hline
      100439 & Martinez, Selene & semartinez@fi.uba.ar \\ \hline
      padron & apellido  nombre & mail \\ \hline
      padron & apellido  nombre & mail \\ \hline
      padron & apellido  nombre & mail \\ \hline
      padron & apellido  nombre & mail \\ \hline
  	\end{tabular}
    \vfill
    \vfill
\end{titlepage}

\tableofcontents % Índice general
\newpage

\section{Introducción}\label{sec:intro}
El presente informe reune la documentación de la solución del primer trabajo práctico de la materia Algoritmos y Programación III que consiste en desarrollar una aplicación de un sistema detección de covid en Pharo utilizando los conceptos del paradigma de la orientación a objetos vistos hasta ahora en el curso.

\section{Supuestos}\label{sec:supuestos}
% Deberá contener explicaciones de cada uno de los supuestos que el alumno haya tenido que adoptar a partir de situaciones que no estén contempladas en la especificación.

Para la resolución del trabajo práctico tuve en cuenta los siguientes supuestos:
\begin{enumerate}
  \item Defecto puede ser un diagnóstico del sistema cuando se ingresa por primera vez a una persona.
  \item Si una persona que es personal esencial se vacuna su estado pasa a ser Vacunado.
  \item Ninguna persona con sintomas puede tener un diagnostico que no sea Sospechoso o Positivo.
  \item Un colegio puede tener clases presenciales aunque no tenga burbujas.
  \item Si una persona tiene un contacto Estrecho con un Sospechoso el diagnostico 'Contacto Estrecho Sospechoso' se superpone a todos los diagnosticos menos a:
  \begin{itemize}
      \item Contacto Estrecho Positivo
      \item Positivo
      \item Sospechoso
  \end{itemize}
  \item Si una persona tiene un contacto estrecho con un positivo el diagnostico 'Contacto Estrecho Positivo' se superpone a todos los diagnosticos menos a:
    \begin{itemize}
      \item Positivo
      \item Sospechoso
  \end{itemize}
  \item Sospechoso solo puede cambiar de diagnostico a Positivo.
  \item Una persona con covid positivo no cambia de diagnostico, solo se le agregan síntomas.
  
  Items 7 y 8 van a modificarse en futuras implementaciones a medida que pierdan sintomas y/o pasen los días de aislamiento correspondientes.
  
   
\end{enumerate}






\section{Diagramas de clase}\label{sec:diagramasdeclase}


\begin{figure}[H]
\centering
\includegraphics[width=0.8\textwidth]{AlgovidCompuestoPorPersonas.png}
\caption{\label{fig:class01}Diagrama Programa AlgoVid y su relación con la clase Persona.}
\end{figure}

\begin{figure}[H]
\centering
\includegraphics[width=0.8\textwidth]{dc_AlgoVidCompuestoPorColegios.png}
\caption{\label{fig:class02}Diagrama del programa AlgoVid y su relación con la clase Colegio. A su vez muestra la relación entre Colegio y Burbuja.}
\end{figure}

\begin{figure}[H]
\centering
\includegraphics[width=0.8\textwidth]{dc_AlgovidCompuestoPorBurbujas.png}
\caption{\label{fig:class03}Diagrama Programa AlgoVid y su relación con la clase Burbuja. Además muestra la relación entre Burbuja <-> su estado y Burbuja <-> Personas}
\end{figure}

\begin{figure}[H]
\centering
\includegraphics[width=1.1\textwidth]{dc_Estado_Herencia.png}
\caption{\label{fig:class04}Diagrama de la clase Estado y su relación con sus hijos.
}
\end{figure}
Estado le delega a las clases abstractas Habilitado/NoHabilitado la implementación de setHabilitacion.\hfill

Vacunadx, Defecto, PersonalEsencial,NoVacunado y CovidNegativo implementan el método setEstado. Por el contrario, la clase abstracta Aislado se lo delega a sus hijos.

\begin{figure}[H]
\centering
\includegraphics[width=1.1\textwidth]{dc_Aislado.png}
\caption{\label{fig:class05}Diagrama de Aislado, incluye clase madre y clases hijas. Las clases hijas implementan el metodo setEstado. Las clases Sospechoso, Positivo y Estrecho Positivo hacen un override del método actualizarEstadoPorburbujaPinchada()}
\end{figure}

\begin{figure}[H]
\centering
\includegraphics[width=1.1\textwidth]{dc_Protocolo_Depende_De_Estado.png}
\caption{\label{fig:class06}Diagrama de estado sobre cómo se relaciona el Estado y el protocolo. Por defecto obtenerProtocolo devuelve ProtoloNoPositivo, si la subclase de estado es Positivo entonces devuelve una instancia de ProtocoloPositivo}
\end{figure}

\section{Detalles de implementación}\label{sec:implementacion}

% Explicaciones sobre la implementación interna de algunas clases que consideren que puedan llegar a resultar interesantes.




\subsection{Colecciones en AlgoVid}
Dado que algoVid almacena varias personas, burbujas y colegios, me decanté por usar una Ordered Collection para los tres atributos

\subsection{Protocolo sin Herencia}
ProtocoloNoPositivo y Protocolo positivo entienden el mismo mensaje: 'activarProtocoloa: Burbuja'.


Persona le pide el protocolo a diagnóstico -> Diagnostico a Estado -> estado al obtenerProtocolo devuelve ProtocoloNoPositivo si es distinto a Positivo. Positivo devuelve un ProtocoloPositivo debido al override.

No necesito usar herencia debido a que ambas clases entienden el mismo mensaje, tampoco necesito que ambas clases compartan la interfaz ya que si bien comparten la firma de un método, solo tienen uno y sus ejecuciones son distintas.
\subsection{Habilitado/NoHabilitado}
Estas son dos clases abstractas pero que definen la implementacion del método: 
\verb   setHabilitación() \newline
este mensaje establece el estado true o false del atributo habilitadoCirculación de Estado.

Clases que heredan de NoHabilitado:
\begin{itemize}
\item Aislado
\item NoVacunado
\item Defecto
\end{itemize}
ninguna de ellas tiene permiso para circular
\newline

Clases que heredan de Habilitado
\begin{itemize}
\item Vacunadx
\item PersonalEsencial
\item CovidNegativo
\end{itemize}
todas tienen permiso para circular
\newline

\subsection{Aislado}
Es una clase que no puede ser PersonalEsencial, por más que se vacune no puede salir -> su estado no cambia porque con Vacuna aún contagian/Pueden Contagiarse,  hereda de NoHabilitado. Si se le envía el mensaje deRiesgo se devuelve a sí misma debido a que por más que puedan ser de riesgo, están en aislamiento por sintomas/contacto estrecho.
Las personas de riesgo se mantienen siendo de riesgo.

Acontinuación se enumerarán clases que heredan de Aislado y sus particularidades

\subsubsection{PersonaDeRiesgo}
No puede ser Esencial porque corre peligro. Es de riesgo aunque se vacune. Por eso hereda de aislado. 

\subsubsection{EstrechoPositivo}
Esta clase además de Heredar de Aislado utiliza un override en actualizarPor: para evitar que pase de contacto estrecho positivo a contacto estrecho sospechoso.

\subsubsection{EstrechoSospechoso}
Esta clase si puede cambiar por contacto estrecho

\subsubsection{PuedeTenerSintomas}
A diferencia de las dos clases anteriores que ante el primer sintoma cambian automáticamente a Sospechoso, estas pueden agregar varios sintomas al mismo tiempo. No se debe a no ser que sea parte de una actualización de estado ya que se podría tener un positivo con menos de 4 sintomas y ninguno habitual. Como sospechoso nunca va a utilizar conSintomas: sintomas, se lo deja sin implementar.
Las dos clases siguientes heredan de PuedeTenerSintomas.


\subsubsection{Sospechoso}
actualizarPorBurbujaPinchada no actualiza a EstrechoPositivo porque es más riesgoso ser Sospechoso que un contacto estrecho. Lo mismo sucede con actualizaPor: contactoEstrechoPositivo/Sospechoso.

Cuando agregamos un sintoma a Sospechoso primero lo va a agregar a los sintomas que son una OrderedCollection.
Luego se pregunta haySintomasHabituales y sintomas size = 3 o sintomas size >4.
Esto es porque si tiene tres sintomas y por lo menos uno es habitual de covid, o tiene mas de tres sintomas, tiene actualizar el diagnostico a Positivo. Si no se deveulve a si mismo.

\subsubsection{Positivo}
actualizarPorBurbujaPinchada también se sobreescribe así como actualizar:Por ya que Positivo es el rango más alto de contagio,
Cuando agregamos sintoma simplemente lo agrega con add: sintoma a la OrderedCollection de los sintomas.

ContactoEstrecho devuelve estrechoPositivo.






\subsection{Burbuja}

La clase Burbuja tiene un atributo llamado estado. Este atributo por defecto es BurbujaSana, si se pincha la burbuja cambia a BurbujaPinchada.

Cuando la Burbuja recibe el mensaje pincharBurbuja delega al estado el cambio del mismo y la notificación a integrantes ya que BurbujaSana y BurbujaPinchada entienden los mismos mensajes, solo cambia su implementacion,



\subsection{Conflictos que se me presetaron}
A la hora de desarrollar el tp me generaba conflicto que persona se transformara en contactoEstrechoPositivo y la burbuja automaticamente tenia que pincharse. No me gustaba la idea de pasarle la burbuja a mi diagnostico para que se encargue. \newline
Lo que buscaba era una forma de poder decirle a la persona 'pincha tu burbuja' se lo delegue a la burbuja.
Pensé en crear una clase PersonaPositiva y que sepa pinchar la burbuja si fuese positiva, si no, que no haga nada. Pero surgía un   nuevo problema: Cómo actualizar el diagnostico. No quiero que la persona se encargue de saber cómo actualizar su estado. Quiero delegarlo.\newline
Luego de mucho pensar llegué a esta aproximación:\newline

Una persona tiene un diagnostico que es instancia de Diagnóstico, una burbuja que es instancia de Burbuja, un protocolo que es ProtocoloNoPositivo/ProtocoloPositivo dependiendo del diagnostico.
La persona le dice al diagnostico agregar sintomas, el diagnostico se lo pasa al estado. Todos los estados menos el Sospechoso y Positivo cambian automaticamente a Sospechoso al agregar un sintoma.
Por defecto se crea un Diagnostico con un estado Defecto que tiene un protocolo no positivo.
Positivo es la única clase con un ProtocoloPositivo.
Al agregar un sintoma se actualiza el protocolo (por si es positivo). El protocolo le envia a la burbuja  
activarProtocoloPorSintoma: sintoma para que se pinche si es un ProtocoloPositivo. Si no lo es no hace nada. \newline



Por ahora tengo una OrderedCollection de sintomasHabituales, tal vez estaria bueno que sea una OrderedCollection de Sintomas y que una clase Sintoma sepa si es Habitual o NoHabitual como Protocolo/ Burbuja.

\subsection{Aclaraciones}
Aislado, Habilitado/NoHabilitado/ PuedeTenerSintoma/Estado no deben instanciarse.


\section{Excepciones}\label{sec:excepciones}
% Explicación de cada una de las excepciones creadas y con qué fin fueron creadas.

\begin{description}
\item[Exception]NoPudoEncontrarPersona, se lanza cuándo se le pide a algovid que busque una persona que no esta instanciada o sin nombre.
\item[Excepcion] NoPudoEncontrarBurbuja se lanza cuándo se le pide a algovid que busqe una burbuja no instanciada o sin nombre
\item[Excepcion] NoPudoEncontrarColegio se lanza cuando se le pide a algovid que busque un colegio no instanciado o sin nombre. 


\end{description}
\subsection{Manejo de Excepciones}

Decidí hacer tres excepciones separadas para que la persona sepa bien dónde está fallando ya que, al hacer algovid agregar: nombreBurbuja a: nombreColegio, si no encuentra alguno de los dos lanzaría notFound  pero no sabría cuál de los dos no pudo encontrar.
Cuando a colegio se le envía el mensaje clasesPresenciales, este devuelve true o false dependiendo del porcentaje de burbujas pinchadas. Para calcular el porcentaje de burbujas pinchadas la clase divide BurbujasPinchadas sobre BurbujasTotales. Si el colegio no tiene burbujas va a lanzar ZeroDivisionError. Tomé la decisiónn de manejar esta excepción haciendo que si no hay burbujas en colegio el porcentajeDeBurbujasPinchadas tiene que devolver burbujasTotales, es decir 0.


\section{Diagramas de secuencia}\label{sec:diagramasdesecuencia}
% Mostrar las secuencias interesantes que hayan implementado. Pueden agregar texto para explicar si algo no queda claro.
\begin{figure}[H]
\centering
\includegraphics[width=1.1\textwidth]{ds_AgregarSintomaASospechosoCon3SitomasLoVuelvePositivo.png}
\caption{\label{fig:seq07}Si agregaSintoma a Sospechoso con tres sintomas se vuelve Positivo.}
\end{figure}

Luego de que diagnostico actualice su estado de Sospechoso a Positivo, Persona va a actualizar su protocolo, pidiendoselo al diagnostico, y por último activa el protocolo correspondiente al diagnostico.
En este caso al ser Positivo el protocolo debería cambiar el estado de todos en la burbuja a estrecho Positivo y pinchar la misma.


\begin{figure}[H]
\centering
\includegraphics[width=0.9 \textwidth]{ds_AgregarSintomaHabitualASospechosoConUnSintomaLoMantieneSospechoso.png}
\caption{\label{fig:seq08}Si sospechoso tiene dos o menos sintomas, aunque se le adicione un sintoma habitual, su diagnostico va a ser Sospechoso}
\end{figure}
\begin{figure}[H]
\centering
\includegraphics[width=0.5\textwidth]{ds_personaActivaProtocoloPositivo.png}
\caption{\label{fig:seq09}Persona Positiva activa protocolo y pincha su burbuja}
\end{figure}

\begin{figure}[H]
\centering
\includegraphics[width=1\textwidth]{ds_actualizar_por_burbuja_pinchada.png}
\caption{\label{fig:seq10}Diagrama de secuencia dependiendo de que estado es es el diagnostico que se obtiene luego de pinchar la burbuja}
\end{figure}



\end{document}
